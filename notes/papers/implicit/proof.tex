
Our goal is to prove that we only compute with well-typed terms, in the sense
of {\Core}. This guarantees the existence of normal forms, and hence, ensure
decidability of type checking.

The proof will be done in two stages: first we prove soundness in the absence
of constraint simplification, and then we prove that constraint simplification
is sound.

\subsection{Without constraint solving}

There are a number of things we need to prove: that type checking preserves
well-formed signatures, that it produces well-typed terms, that conversion
checking is sound, and that new signatures respect the old signatures.
Unfortunately these properties are all interdependent, so we cannot prove them
separately. Instead we prove them all in one fell swoop. First we define what
we mean by a signature extension.

\begin{definition}[Signature extension]
    We say that $\Sigma'$ {\em extends} $\Sigma$ if and only if
    \[\begin{array}{lcl}
	\IsTypeCS \Sigma \Gamma A & \implies & \IsTypeCS {\Sigma'} \Gamma A \\
	\HasTypeCS \Sigma \Gamma M A & \implies & \HasTypeCS {\Sigma'} \Gamma M A \\
	\EqualTypeCS \Sigma \Gamma A B & \implies & \EqualTypeCS {\Sigma'} \Gamma A B \\
	\EqualCS \Sigma \Gamma M N A & \implies & \EqualCS {\Sigma'} \Gamma M N A \\
    \end{array}\]
\end{definition}

Note that this definition admits both simple extensions--adding a new
constant--and refinement, where we give a definition to a constant. This is
expressed by the following two lemmas.

\begin{lemma}[Signature weakening] \label{lemWeakenSig}
    If $\Sigma' = \Ext \Sigma {c:A}$ and $\IsSigCS {\Sigma'}$ then $\Extends
    {\Sigma'} \Sigma$.
\end{lemma}

\begin{lemma}[Signature refinement] \label{lemRefineSig}
    Giving a definition to a constant in a signature is an extension of the
    signature. More precisely, if
    \begin{itemize}
	\item $\HasTypeCS {\Sigma_1} {} M A$
	\item $\Sigma ~ = ~ \Ext {\Sigma_1} \Ext {\MetaDecl c A} \Sigma_2$
	\item $\Sigma' ~ = ~ \Ext {\Sigma_1} \Ext {\IMetaDecl c A M} \Sigma_2$
    \end{itemize}
    then $\Extends {\Sigma'} \Sigma$.
\end{lemma}

\begin{proof}[Proofs]
    In both lemmas any derivation using $\Sigma$ is immediately valid also with $\Sigma'$.
\end{proof}

To express the soundness of conversion checking we need to define when a
constraint is well-formed.

\begin{definition}[Well-formed constraint]
    The rules for well-formed constraints are
    \[
	\infer{ \ValidConstr \Sigma {\TypeConstr A B}}
	{\begin{array}{l}
	    \IsTypeCS \Sigma \Gamma A \\
	    \IsTypeCS \Sigma \Gamma B 
	\end{array}}
    \qquad
	\infer{ \ValidConstr \Sigma {\TermConstr M N A}}
	{\begin{array}{l}
	    \HasTypeCS \Sigma \Gamma M A \\
	    \HasTypeCS \Sigma \Gamma N A \\
	\end{array}}
    \]
\end{definition}

Well-formedness extends in the obvious way to sets of constraints.

\begin{theorem}[Soundness of type checking] \label{thmSoundNoCs}
    Type checking produces well-typed terms, conversion checking produces
    well-formed constraints and if no constraints are produced, the conversion
    is valid in {\Core}. Also, all rules produce well-formed extensions of the
    signature.
    \begin{itemize}

	% is type
	\item
	    $\begin{array}[t]{l}
		\ExplicitJudgement \Sigma {\IsType e A} {\Sigma'}
		~ \wedge ~ \IsCtxCS \Sigma \Gamma \\
		{} \implies \Extends {\Sigma'} {\Sigma}
		~ \wedge ~ \IsTypeCS {\Sigma'} \Gamma A
	    \end{array}$

	% check type
	\item
	    $\begin{array}[t]{l}
		\ExplicitJudgement \Sigma {\CheckType e A M} {\Sigma'}
		~ \wedge ~ \IsTypeCS \Sigma \Gamma A \\
		{} \implies \Extends {\Sigma'} {\Sigma}
		~ \wedge ~ \HasTypeCS {\Sigma'} \Gamma M A
	    \end{array}$

	% infer type
	\item
	    $\begin{array}[t]{l}
		\ExplicitJudgement \Sigma {\InferType e A M} {\Sigma'}
		~ \wedge ~ \IsCtxCS \Sigma \Gamma \\
		{} \implies \Extends {\Sigma'} {\Sigma}
		~ \wedge ~ \HasTypeCS {\Sigma'} \Gamma M A
	    \end{array}$

	% equal types
	\item
	    $\begin{array}[t]{l}
		\ExplicitJudgement \Sigma {\EqualType A B \Cs} {\Sigma'}
		~ \wedge ~ \IsTypeCS \Sigma \Gamma A
		~ \wedge ~ \IsTypeCS \Sigma \Gamma B
		\\
		{} \implies \Extends {\Sigma'} {\Sigma}
		~ \wedge ~ \ValidConstr {\Sigma'} \Cs
		~ \wedge ~ (\Cs = \emptyset \implies \EqualTypeCS {\Sigma'} \Gamma A B)
	    \end{array}$

	% equal terms
	\item
	    $\begin{array}[t]{l}
		\ExplicitJudgement \Sigma {\Equal M N A \Cs} {\Sigma'}
		~ \wedge ~ \HasTypeCS \Sigma \Gamma M A
		~ \wedge ~ \HasTypeCS \Sigma \Gamma N A
		\\
		{} \implies \Extends {\Sigma'} {\Sigma}
		~ \wedge ~ \ValidConstr {\Sigma'} \Cs
		~ \wedge ~ (\Cs = \emptyset \implies \EqualCS {\Sigma'} \Gamma M N A)
	    \end{array}$

    \end{itemize}

    The statements for weak head normal form conversion ($\EqualWhnf M N A
    \Cs$) and term sequence conversion ($\Equal {\bar M} {\bar N} \Delta \Cs$)
    are equivalent to that of term conversion.
\end{theorem}

\begin{proof}
    By induction on the derivation. Some interesting cases:
    \begin{itemize}

	\item {\em (TODO: Not so interesting)} In the type conversion case for
	function spaces where the domains produce constraints, we have to use a
	substitution lemma from {\Core}, stating that if $\IsTypeC {\Ext \Gamma
	x:A} B$ and $\HasTypeC \Gamma M A$, then $\IsTypeC \Gamma {\Subst B x
	M}$.

	\item In the term conversion case where the terms are weak head
	normalised we need subject reduction for weak head normalisation.

	\item When checking conversion of terms with the same head we need an
	inversion principle for application. If $\HasTypeC \Gamma M {\PI x A
	B}$ and $\HasTypeC \Gamma {M \, N} {B_1}$ then $\HasTypeC \Gamma N A$.

	\TODO{this will change if we move to Conor zipping}

	\item The most interesting case is the meta variable instantiation case.

    \end{itemize}
\end{proof}

% \begin{lemma}[Well-formed signatures] \label{lemValidSig}
%     Type checking preserves well-formed signatures. More precisely, for $J$
%     ranging over the type checking and conversion checking judgements we
%     have
%     \[{\ExplicitJudgement \Sigma J {\Sigma'}} \, \mathrel{\wedge} \,
% 	{\IsSigCS \Sigma}  ~ \implies ~ \IsSigCS {\Sigma'}
%     \]
% \end{lemma}
% \begin{proof}
%     By induction on the derivation of $\ExplicitJudgement \Sigma J {\Sigma'}$.
%     We only have to check the cases where we change the signature, i.e. when we
%     add a meta variable or guarded constant or when we instantiate a meta
%     variable. In the case where we add a constant with a non-empty guard we can
%     ignore the body, since it is not visible in {\Core}. The only interesting
%     case is the meta variable instantiation case.
% \end{proof}

% \begin{lemma} \label{lemExtendSig}
%     If $\ExplicitJudgement \Sigma J {\Sigma'}$ then $\Sigma'$ extends $\Sigma$.
% \end{lemma}

% In order to guarantee type safety we need soundness of conversion checking.

% \begin{lemma}[Soundness of term conversion checking, no constraints] \label{lemConvSound}
%     If conversion checking of two terms does not produce any constraints, then
%     they are convertible in {\Core}. More precisely if $\IsTypeCS \Sigma \Gamma A$,
%     $\IsTypeCS \Sigma \Gamma B$, $\HasTypeCS \Sigma \Gamma M A$, and
%     $\HasTypeCS \Sigma \Gamma N A$, then
%     \[\begin{array}{lcl}
% 	\ExplicitJudgement \Sigma {\EqualType A B \emptyset} {\Sigma'}
% 	    & \implies & \EqualTypeCS {\Sigma'} \Gamma A B \\
% 	\ExplicitJudgement \Sigma {\Equal M N A \emptyset} {\Sigma'}
% 	    & \implies & \EqualCS {\Sigma'} \Gamma M N A \\
%     \end{array}\]
% \end{lemma}

% \begin{proof}
%     By induction on the derivation using the fact that weak head normalisation
%     preserves conversion in {\Core}. The only case where something
%     interesting happens is the meta variable instantiation case where we have
%     to prove $\EqualCS {\Sigma'} \Gamma {\alpha\,\bar x} M A$, where $\Sigma'$
%     defines $\alpha = \LAM {\bar x} M$. By Lemma ~ \ref{lemValidSig} we know
%     that $\Sigma'$ is well-formed, and so the proof is by unfolding $\alpha$
%     and performing $\beta$-reducing the result.
% \end{proof}

% \begin{theorem}[Type-safety] \label{thmTypeSafety}
%     Terms produced by type checking are type correct in {\Core} regardless of any
%     constraints.

%     \[\begin{array}{lcl}
% 	{\ExplicitJudgement \Sigma {\IsType e A} {\Sigma'}}
% 	& \implies & {\IsTypeCS {\CoreSig{\Sigma'}} \Gamma A} \\
% 	{\ExplicitJudgement \Sigma {\CheckType e A M} {\Sigma'}}
% 	\, \wedge \, {\IsTypeCS {\CoreSig \Sigma} \Gamma A}
% 	& \implies & \HasTypeCS {\CoreSig{\Sigma'}} \Gamma M A \\
% 	{\ExplicitJudgement \Sigma {\InferType e A M} {\Sigma'}}
% 	& \implies & \HasTypeCS {\CoreSig{\Sigma'}} \Gamma M A
%     \end{array}\]
% \end{theorem}

% \begin{proof}
%     The proof is straighforward by induction on the typing derivation. In the
%     case of the conversion rule, if the types are convertible without producing
%     any constraints we rely on the soundness of the conversion checker
%     (Lemma ~ \ref{lemConvSound}). If there are constraints the resulting term
%     is a fresh constant that is well-typed by definition.
% \end{proof}

Since well-typed terms in {\Core} have weak head normal forms we get the
existence of weak head normal forms for type checked terms.

\begin{corollary}
    Type checked terms have weak head normal forms.
\end{corollary}

Once we have weak head normal forms it is easy to verify that the rules
describe a decidable type checking algorithm.

\begin{corollary}
    Type checking is decidable.
\end{corollary}

\subsection{Constraint solving}

% \begin{lemma}[Soundness of term conversion checking] \label{thmConvSound}
%     If two terms are checked convertible relative to a set of constraints, then
%     any solution to these constraints makes the terms convertible in {\Core}.
%     More precisely, if
%     \begin{itemize}
% 	\item \( \HasTypeCS \Sigma \Gamma M A ~ \wedge ~
% 		 \HasTypeCS \Sigma \Gamma N A
% 	      \)
% 	\item $\ExplicitJudgement \Sigma {\Equal M N A \Cs} {\Sigma'}$
% 	\item $ \Extends {\Sigma_1} {\Sigma'}$
% 	\item $\ExplicitJudgement
% 		{\Sigma_1}
% 		{\CheckConstr \Cs \emptyset}
% 		{\Sigma_2}
% 	      $
%     \end{itemize}
%     then \( \EqualCS {\Sigma_2} \Gamma M N A \)
% \end{lemma}

% \begin{proof}
%     The interesting case is meta variable instantiation.
% \end{proof}

% The soundness of type conversion checking is expressed in the same way.

% \begin{lemma}[Soundness of type conversion checking]
%     If
%     \begin{itemize}
% 	\item \( \IsTypeCS \Sigma \Gamma A ~ \wedge ~
% 		 \IsTypeCS \Sigma \Gamma B
% 	      \)
% 	\item $\ExplicitJudgement \Sigma {\EqualType A B \Cs} {\Sigma'}$
% 	\item $ \Extends {\Sigma_1} {\Sigma'}$
% 	\item $\ExplicitJudgement
% 		{\Sigma_1}
% 		{\CheckConstr \Cs \emptyset}
% 		{\Sigma_2}
% 	      $
%     \end{itemize}
%     then \( \EqualTypeCS {\Sigma_2} \Gamma A B \)
% \end{lemma}

% \begin{corollary} \label{corConvSound}
%     If $\ExplicitJudgement \Sigma {\EqualType A B \emptyset} {\Sigma'}$ then
%     $\EqualTypeCS {\Sigma'} \Gamma A B$.
% \end{corollary}

% \begin{theorem}[Soundness]
%     If all meta variables are solved and all guards are empty, then we are fine. More precisely, if
%     \begin{itemize}
% 	\item $\ExplicitJudgement \Sigma {\CheckType e A M} {\Sigma'}$
% 	\item ${\ExplicitJudgement {\Sigma'} \Simplify {\Sigma''}}$
% 	\item ${\Implements {\Sigma''} {\Sigma}}$ (everything new in $\Sigma''$ has been {\em solved})
% 	\item ${\ExplicitJudgement {\Sigma''} {\Inline M {\hat M}} {\Sigma''}}$
%     \end{itemize}
%     then $\HasTypeCS \Sigma \Gamma {\hat M} A$.
% \end{theorem}
% \begin{proof}
%     Follows from Theorem \ref{thmTypeSafety} and Lemma \ref{lemExtendSig} and
%     some property of inlining.
% \end{proof}

\begin{theorem}[Completeness]
    We are only complete in the sense that if there exists a solution, we find
    an approximation of it.
\end{theorem}
\begin{proof}
    To give the proof we would have to spell out when type checking fails in
    more detail.
\end{proof}

