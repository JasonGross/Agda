\documentclass[11pt]{article}

%% ODER: format ==         = "\mathrel{==}"
%% ODER: format /=         = "\neq "
\makeatletter

\usepackage{amstext}
\usepackage{amssymb}
\usepackage{stmaryrd}
\DeclareFontFamily{OT1}{cmtex}{}
\DeclareFontShape{OT1}{cmtex}{m}{n}
  {<5><6><7><8>cmtex8
   <9>cmtex9
   <10><10.95><12><14.4><17.28><20.74><24.88>cmtex10}{}
\DeclareFontShape{OT1}{cmtex}{m}{it}
  {<-> ssub * cmtt/m/it}{}
\newcommand{\texfamily}{\fontfamily{cmtex}\selectfont}
\DeclareFontShape{OT1}{cmtt}{bx}{n}
  {<5><6><7><8>cmtt8
   <9>cmbtt9
   <10><10.95><12><14.4><17.28><20.74><24.88>cmbtt10}{}
\DeclareFontShape{OT1}{cmtex}{bx}{n}
  {<-> ssub * cmtt/bx/n}{}
\newcommand{\tex}[1]{\text{\texfamily#1}}	% NEU

\newcommand{\Sp}{\hskip.33334em\relax}

\newlength{\lwidth}\setlength{\lwidth}{4.5cm}
\newlength{\cwidth}\setlength{\cwidth}{8mm} % 3mm

\newcommand{\Conid}[1]{\mathit{#1}}
\newcommand{\Varid}[1]{\mathit{#1}}
\newcommand{\anonymous}{\kern0.06em \vbox{\hrule\@width.5em}}
\newcommand{\plus}{\mathbin{+\!\!\!+}}
\newcommand{\bind}{\mathbin{>\!\!\!>\mkern-6.7mu=}}
\newcommand{\sequ}{\mathbin{>\!\!\!>}}
\renewcommand{\leq}{\leqslant}
\renewcommand{\geq}{\geqslant}
\newcommand{\NB}{\textbf{NB}}
\newcommand{\Todo}[1]{$\langle$\textbf{To do:}~#1$\rangle$}

\makeatother




\usepackage{ucs}
\usepackage[utf8x]{inputenc}
\usepackage{autofe}
\usepackage{color}

\usepackage{amsthm}
\newtheorem{theorem}{Theorem}[section]
\newtheorem{lemma}[theorem]{Lemma}
\newtheorem{corollary}[theorem]{Corollary}
\newtheorem{definition}[theorem]{Definition}

% Enables greek letters in math environment
\everymath{\SetUnicodeOption{mathletters}}
\everydisplay{\SetUnicodeOption{mathletters}}

% This makes sure that local glyph overrides below are
% chosen.
\DeclareUnicodeOption{localDefs}
\SetUnicodeOption{localDefs}

% For some reason these macros need to be defined.
\newcommand{\textmu}{$\mu$}
\newcommand{\textnu}{$\nu$}

% This character doesn't seem to be defined by ucs.sty.
\DeclareUnicodeCharacter{"21A6}{\ensuremath{\mapsto}}


\newlength\TODOWidth

% Administrative
\newcommand\TODO[1]{
\par
\vspace{1mm}
\fbox{
\setlength\TODOWidth{\textwidth}
\addtolength\TODOWidth{-60pt}
\begin{tabular}{p{\TODOWidth}}
    TODO: #1
\end{tabular}}
}
% \newcommand\TODO[1]{\par{
%     \setbox0=\vbox{{\small \parbox{35mm}{#1}}} %{\parbox{3.5cm}{{\small #1}}}
%     \hskip-50mm\copy0
%     \vskip-\ht0
%     \vskip-2mm
% }\par
% }

\newcommand\Core{{\sf\bf MLF}}

% Misc
\renewcommand\Or{~~|~~}
\newcommand\C{\mathcal C}
\newcommand\Pair[1]{\langle#1\rangle}

% Syntax
\newcommand\SET{\mathsf{Set}}
\newcommand\EL{\mathsf{El}\,}
\newcommand\unify{\mathsf{unify}}
\newcommand\inscope{\mathsf{InScope}}
\newcommand\PI[2]{(#1:#2)\to{}}
\newcommand\LAM[1]{\lambda #1.{}}

\newcommand\Subst[3]{ {#1} [ {#3} / {#2} ] }
\newcommand\SubstD[2]{{#1} [ {#2} ] }

% Judgement forms

% Core
\newcommand\IsSigCS[1]{{} \vdash_{#1}}
\newcommand\IsCtxCS[2]{{#2} \vdash_{#1}}
\newcommand\IsTypeCS[3]{{#2} \vdash_{#1}#3 ~ \mathbf{type}}
\newcommand\HasTypeCS[4]{{#2} \vdash_{#1} {#3} : {#4}}
\newcommand\CheckTypeCS[4]{{#2} \vdash_{#1}#3\uparrow#4}
\newcommand\InferTypeCS[4]{{#2} \vdash_{#1}#3\downarrow#4}
\newcommand\EqualTypeCS[4]{{#2} \vdash_{#1}#3=#4}
\newcommand\EqualCS[5]{{#2} \vdash_{#1}#3=#4:#5}

\newcommand\IsCtxC[1]{\IsCtxCS{}{#1}}
\newcommand\IsTypeC[2]{\IsTypeCS{}{#1}{#2}}
\newcommand\HasTypeC[3]{\HasTypeCS{}{#1}{#2}{#3}}
\newcommand\CheckTypeC[3]{\CheckTypeCS{}{#1}{#2}{#3}}
\newcommand\InferTypeC[3]{\InferTypeCS{}{#1}{#2}{#3}}
\newcommand\EqualTypeC[3]{\EqualTypeCS{}{#1}{#2}{#3}}
\newcommand\EqualC[4]{\EqualCS{}{#1}{#2}{#3}{#4}}

% With metas
\newcommand\Cs{\mathcal{C}}

\newcommand\Ext[1]{{#1},\,{}}

\newcommand\ExplicitJudgement[3]{
    \langle {#1} \rangle ~ {#2} \longrightarrow \langle {#3} \rangle
}

\newcommand\MetaDecl[2]{{#1} : {#2}}

\newcommand\IMetaDecl[3]{{#1} : {#2} = {#3}}

\newcommand\ConstDecl[4]{{#1} : {#2} = {#3} ~\mathbf{when}~ {#4}}

\newcommand\AddMeta[2]{\mathsf{AddMeta}({#1} : {#2})}
\newcommand\AddConst[4]{\mathsf{AddConst}({#1} : {#2} = {#3}~\mathbf{when}~{#4})}
\newcommand\InstMeta[2]{{#1} := {#2}}
\newcommand\LookupMeta[2]{\mathsf{LookupMeta}({#1} = {#2})}
\newcommand\LookupConst[2]{\mathsf{LookupConst}({#1} = {#2})}
\newcommand\LookupType[2]{\mathsf{Lookup}({#1} : {#2})}
\newcommand\InScope[2]{\mathsf{InScope}_{#1}({#2})}
\newcommand\Simplify{\mathsf{Simplify}}
\newcommand\WithSig[1]{\langle {#1} \rangle \,}
\newcommand\UpdateGuard[2]{\mathsf{UpdateGuard}( {#1},\, {#2} )}
\newcommand\LookupClause[3]{\mathsf{Lookup}({#1}~{#2} = {#3})}
\newcommand\Match[3]{\mathsf{Match}({#1}\,{#2} = {#3})}
\newcommand\Guarded[1]{\mathsf{Guarded}({#1})}
\newcommand\Uninstantiated[1]{\mathsf{Uninstantiated}({#1})}
\newcommand\Blocked[1]{\mathsf{Blocked}({#1})}
\newcommand\Inline[2]{{#1} \to_{\mathit{inline}} {#2}}

\newcommand\IsTypeCtx[3]{{#1} \vdash {#2} ~ \mathbf{type} ~ \leadsto {#3}}
\newcommand\CheckTypeCtx[4]{{#1} \vdash {#2} \uparrow {#3} \leadsto {#4}}
\newcommand\InferTypeCtx[4]{{#1} \vdash {#2} \downarrow {#3} \leadsto {#4}}
\newcommand\EqualTypeCtx[4]{{#1} \vdash {#2} = {#3} \leadsto {#4}}
\newcommand\EqualCtx[5]{{#1} \vdash {#2} = {#3} : {#4} \leadsto {#5}}
\newcommand\EqualWhnfCtx[5]{{#1} \vdash {#2} \doteq {#3} : {#4} \leadsto {#5}}
% \newcommand\EqualWhnfCtx[5]{{#1} \vdash {#2} \mathrel{\stackrel{\mathit{whnf}}=} {#3} : {#4} \leadsto {#5}}

\newcommand\TypeConstrCtx[3]{\EqualTypeC {#1} {#2} {#3}}
\newcommand\TermConstrCtx[4]{\EqualC {#1} {#2} {#3} {#4}}
\newcommand\TypeConstr[2]{\TypeConstrCtx \Gamma {#1} {#2}}
\newcommand\TermConstr[3]{\TermConstrCtx \Gamma {#1} {#2} {#3}}
\newcommand\ValidConstr[2]{\vdash_{#1} {#2} ~ \mathbf{ok}}
\newcommand\TrueConstr[2]{\vdash_{#1} {#2} ~ \mathbf{true}}

\newcommand\IsType[2]{\IsTypeCtx \Gamma {#1} {#2}}
\newcommand\CheckType[3]{\CheckTypeCtx \Gamma {#1} {#2} {#3}}
\newcommand\InferType[3]{\InferTypeCtx \Gamma {#1} {#2} {#3}}
\newcommand\EqualType[3]{\EqualTypeCtx \Gamma {#1} {#2} {#3}}
\newcommand\Equal[4]{\EqualCtx \Gamma {#1} {#2} {#3} {#4}}
\newcommand\EqualWhnf[4]{\EqualWhnfCtx \Gamma {#1} {#2} {#3} {#4}}
\newcommand\CheckConstr[2]{{#1} \leadsto {#2}}

\newcommand\whnf[2]{{#1} \to_{\mathit{whnf}} {#2}}
\newcommand\Normalise[2]{{#1} \to_{\mathit{nf}} {#2}}

\newcommand\FV[1]{\mathsf{FV}({#1})}
\newcommand\AppSub[2]{{#2}{#1}}

\newcommand\CombinedSig[3]{\mathsf{Combine}({#1},{#2},{#3})}
\newcommand\CoreSig[1]{\left|{#1}\right|}
\newcommand\Implements[2]{{#2} \propto {#1}}

\newcommand\Extends[2]{{#1} ~ \mathit{extends} ~ {#2}}

\newcommand\Rules[2]{
\par~\par
{\setlength\parindent{0mm}
    {\em #1}{\small
    \[\begin{array}{c}
	#2
    \end{array}\]
    }
}}

\newcommand\URules[1]{
{\small\[\begin{array}{c}
    #1
\end{array}\]
}}



\title{Encoding indexed inductive types using the identity type}
\author{Ulf Norell}

\begin{document}
\maketitle
\begin{abstract}
    An indexed inductive-recursive definition (IIRD) simultaneously defines an
    indexed family of sets and a recursive function over this family.  This
    notion is sufficiently powerful to capture essentially all definitions of
    sets in Martin-Löf type theory.

    I show that it is enough to have one particular indexed inductive type,
    namely the intensional identity relation, to be able to interpret all IIRD
    as non-indexed definitions.
    
    The proof is formally verified in Agda.
\end{abstract}

\section{Introduction}

% Describe the current state of affairs

Indexed induction recursion is the thing.

% Indentify gap

Dybjer and Setzer~\cite{dybjer:indexed-ir} show that in an extensional theory
generalised IIRD can be interpreted by restricted IIRD~\cite{dybjer:jsl}. We're
not using an extensional theory though.

\TODO{What's the relation between restricted IIRD and IRD?}

% Fill gap

I improve on this result showing that it is enough to add intensional equality.
The proof is formalised in Agda.

Non-indexed definitions are simpler(?), so if can get away with just adding the
identity type we get simpler meta theory that if we would add indexed
definitions directly.

\section{The Logical Framework}

    Martin-Löf's logical framework~\cite{nordstrom:book} extended with sigma
    types ($\SIGMA x A B$), $\Zero$, $\One$, and $\Two$.

    \TODO{what about $Π$ in Set? Used on the meta level but probably not on the object level.}

    $\HasType {Γ} x A$

    $\IsType {Γ} A$

    $\PI x A B$

    $\SIGMA x A B$

\section{The Identity Type}

There are many versions.

\begin{tabbing}
\qquad\=\hspace{\lwidth}\=\hspace{\cwidth}\=\+\kill
${(==)\mathbin{:}\{\mskip1.5mu \Conid{A}\mathbin{:}\Set\mskip1.5mu\}\to \Conid{A}\to \Conid{A}\to \Set}$\\
${\Varid{refl}\mathbin{:}\{\mskip1.5mu \Conid{A}\mathbin{:}\Set\mskip1.5mu\}(\Varid{x}\mathbin{:}\Conid{A})\to \Varid{x}==\Varid{x}}$
\end{tabbing}
Martin-Löf identity relation, introduced in 1973~\cite{martin-lof:predicative}.
The elimination rule (sometimes called $J$) has the type

\begin{tabbing}
\qquad\=\hspace{\lwidth}\=\hspace{\cwidth}\=\+\kill
${\Varid{elim\char95 ML}\mathbin{:}\{\mskip1.5mu \Conid{A}\mathbin{:}\Set\mskip1.5mu\}(\Conid{C}\mathbin{:}(\Varid{x},\Varid{y}\mathbin{:}\Conid{A})\to \Varid{x}==\Varid{y}\to \Set)\to }$\\
${\phantom{\Varid{elim\char95 ML}\mathbin{:}\mbox{}}((\Varid{x}\mathbin{:}\Conid{A})\to \Conid{C}\;\Varid{x}\;\Varid{x}\;(\Varid{refl}\;\Varid{x}))\to }$\\
${\phantom{\Varid{elim\char95 ML}\mathbin{:}\mbox{}}(\Varid{x},\Varid{y}\mathbin{:}\Conid{A})\;(\Varid{p}\mathbin{:}\Varid{x}==\Varid{y})\to \Conid{C}\;\Varid{x}\;\Varid{y}\;\Varid{p}}$
\end{tabbing}
and the corresponding computation rule is
\begin{tabbing}
\qquad\=\hspace{\lwidth}\=\hspace{\cwidth}\=\+\kill
${\hskip1.00em\relax\Varid{elim\char95 ML}\;\Conid{C}\;\Varid{h}\;\Varid{x}\;\Varid{x}\;(\Varid{refl}\;\Varid{x})\mathrel{=}\Varid{h}\;\Varid{x}}$
\end{tabbing}Paulin identity relation~\cite{pfenning-paulin:inductive-coc}.

\begin{tabbing}
\qquad\=\hspace{\lwidth}\=\hspace{\cwidth}\=\+\kill
${\Varid{elim\char95 P}\mathbin{:}\{\mskip1.5mu \Conid{A}\mathbin{:}\Set\mskip1.5mu\}(\Varid{x}\mathbin{:}\Conid{A})\;(\Conid{C}\mathbin{:}(\Varid{y}\mathbin{:}\Conid{A})\to \Varid{x}==\Varid{y}\to \Set)\to }$\\
${\phantom{\Varid{elim\char95 P}\mathbin{:}\mbox{}}\Conid{C}\;\Varid{x}\;(\Varid{refl}\;\Varid{x})\to (\Varid{y}\mathbin{:}\Conid{A})\;(\Varid{p}\mathbin{:}\Varid{x}==\Varid{y})\to \Conid{C}\;\Varid{y}\;\Varid{p}}$
\end{tabbing}
The corresponding computation rule is
\begin{tabbing}
\qquad\=\hspace{\lwidth}\=\hspace{\cwidth}\=\+\kill
${\hskip1.00em\relax\Varid{elim\char95 P}\;\Varid{x}\;\Conid{C}\;\Varid{h}\;\Varid{x}\;(\Varid{refl}\;\Varid{x})\mathrel{=}\Varid{h}}$
\end{tabbing}\begin{theorem}
    Martin-Löf elimination can be defined in terms of Paulin elimination.
\end{theorem}

\begin{proof}

Trivial.
\begin{tabbing}
\qquad\=\hspace{\lwidth}\=\hspace{\cwidth}\=\+\kill
${\hskip1.00em\relax\Varid{elim\char95 ML}\;\Conid{C}\;\Varid{h}\;\Varid{x}\;\Varid{y}\;\Varid{p}\mathrel{=}\Varid{elim\char95 P}\;\Varid{x}\;(\lambda \Varid{z}\;\Varid{q}.~\Conid{C}\;\Varid{x}\;\Varid{z}\;\Varid{q})\;(\Varid{h}\;\Varid{x})\;\Varid{y}\;\Varid{p}}$
\end{tabbing}\end{proof}

\begin{theorem}
    Paulin elimination can be defined in terms of Martin-Löf elimination.
\end{theorem}

\begin{proof}
    This proof is slightly more involved.

    We first define the substitution rule
\begin{tabbing}
\qquad\=\hspace{\lwidth}\=\hspace{\cwidth}\=\+\kill
${\hskip1.00em\relax\hskip1.00em\relax\Varid{subst}\mathbin{:}\{\mskip1.5mu \Conid{A}\mathbin{:}\Set\mskip1.5mu\}(\Conid{C}\mathbin{:}\Conid{A}\to \Set)\;(\Varid{x},\Varid{y}\mathbin{:}\Conid{A})}$\\
${\hskip1.00em\relax\hskip1.00em\relax\phantom{\Varid{subst}\mathbin{:}\mbox{}}\Varid{x}==\Varid{y}\to \Conid{C}\;\Varid{x}\to \Conid{C}\;\Varid{y}}$\\
${\hskip1.00em\relax\hskip1.00em\relax\Varid{subst}\;\Conid{C}\;\Varid{x}\;\Varid{y}\;\Varid{p}\;\Conid{Cx}\mathrel{=}\Varid{elim\char95 ML}\;(\lambda \Varid{a}\;\Varid{b}\;\Varid{q}.~\Conid{C}\;\Varid{a}\to \Conid{C}\;\Varid{b})\;}$\\
${\hskip1.00em\relax\hskip1.00em\relax\phantom{\Varid{subst}\;\Conid{C}\;\Varid{x}\;\Varid{y}\;\Varid{p}\;\Conid{Cx}\mathrel{=}\Varid{elim\char95 ML}\;\mbox{}}(\lambda \Varid{a}\;\Conid{Ca}.~\Conid{Ca})\;\Varid{x}\;\Varid{y}\;\Varid{p}\;\Conid{Cx}}$
\end{tabbing}
    Now define

\begin{tabbing}
\qquad\=\hspace{\lwidth}\=\hspace{\cwidth}\=\+\kill
${\hskip1.00em\relax\hskip1.00em\relax\Conid{E}\mathbin{:}\{\mskip1.5mu \Conid{A}\mathbin{:}\Set\mskip1.5mu\}(\Varid{x}\mathbin{:}\Conid{A})\to \Set}$\\
${\hskip1.00em\relax\hskip1.00em\relax\Conid{E}\;\Varid{x}\mathrel{=}(\Varid{y}\mathbin{:}\Conid{A})×(\Varid{x}==\Varid{y})}$
\end{tabbing}
    We can prove that any element of \ensuremath{\Conid{E}\;\Varid{x}} is in fact equal to \ensuremath{(\Varid{x},\Varid{refl}\;\Varid{x})}.

\begin{tabbing}
\qquad\=\hspace{\lwidth}\=\hspace{\cwidth}\=\+\kill
${\hskip1.00em\relax\hskip1.00em\relax\Varid{uniqE}\mathbin{:}\{\mskip1.5mu \Conid{A}\mathbin{:}\Set\mskip1.5mu\}(\Varid{x},\Varid{y}\mathbin{:}\Conid{A})\;(\Varid{p}\mathbin{:}\Varid{x}==\Varid{y})\to (\Varid{x},\Varid{refl}\;\Varid{x})==(\Varid{y},\Varid{p})}$\\
${\hskip1.00em\relax\hskip1.00em\relax\Varid{uniqE}\mathrel{=}\Varid{elim}\;(\lambda \Varid{x}\;\Varid{y}\;\Varid{p}.~(\Varid{x},\Varid{refl}\;\Varid{x})==(\Varid{y},\Varid{p}))\;\Varid{refl}}$
\end{tabbing}
\begin{tabbing}
\qquad\=\hspace{\lwidth}\=\hspace{\cwidth}\=\+\kill
${\hskip1.00em\relax\hskip1.00em\relax\Varid{elim\char95 P}\;\Varid{x}\;\Conid{C}\;\Varid{h}\;\Varid{y}\;\Varid{p}\mathrel{=}\Varid{subst}\;(\lambda \Varid{z}.~\Conid{C}\;(π_0\;\Varid{z})\;(π_1\;\Varid{z}))\;}$\\
${\hskip1.00em\relax\hskip1.00em\relax\phantom{\Varid{elim\char95 P}\;\Varid{x}\;\Conid{C}\;\Varid{h}\;\Varid{y}\;\Varid{p}\mathrel{=}\Varid{subst}\;\mbox{}}(\Varid{x},\Varid{refl}\;\Varid{x})\;(\Varid{y},\Varid{p})\;(\Varid{uniqE}\;\Varid{x}\;\Varid{y}\;\Varid{p})\;\Varid{h}}$
\end{tabbing}
    Note that in an impredicative setting there is a simpler proof due to
    Streicher~\cite{streicher:habilitation}.

\end{proof}

\begin{corollary}
    Martin-Löf elimination and Paulin elimination are equivalent.
\end{corollary}

Streicher axiom K. Not valid~\cite{HofmannM:gromru}. Fortunately we don't need it.

In the following we will use Paulin elimination.

\section{Indexed Induction Recursion}

Very fancy and general types~\cite{dybjer:indexed-ir}.

\TODO{How much details from \cite{dybjer:indexed-ir}?}

\subsection{Examples}

Intensional identity relation (Paulin version).

\begin{tabbing}
\qquad\=\hspace{\lwidth}\=\hspace{\cwidth}\=\+\kill
${\mathbf{data}\;(==)\{\mskip1.5mu \Conid{A}\mathbin{:}\Set\mskip1.5mu\}(\Varid{x}\mathbin{:}\Conid{A})\mathbin{:}\Conid{A}\to \Set\;\mathbf{where}}$\\
${\hskip2.00em\relax\Varid{refl}\mathbin{:}\Varid{x}==\Varid{x}}$
\end{tabbing}
The elimination rule for this type is Paulin elimination.

\section{Encoding}

Just add a proof that the index is the right one.

\subsection{Proof}

Simple induction over the code for the indexed type.

\section{Related Work}

Peter and Anton obviously~\cite{dybjer:indexed-ir}.

\section{Conclusions}

This is good stuff.

\bibliographystyle{abbrv}
\bibliography{../../../../bib/pmgrefs}

\end{document}